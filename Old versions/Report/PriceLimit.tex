\documentclass[12pt]{article}

\usepackage{dcolumn}
\usepackage{booktabs}
\usepackage{pdflscape}
\usepackage{graphicx}
\usepackage{placeins}
\usepackage{amsmath}
\usepackage{geometry}
\linespread{1.5}

\begin{document}
\title{Price Limit}
\author{S.Morteza Aghajanzadeh}
\maketitle

Do Stocks in the Tehran market that hit upper price limits typically exhibit  high
returns and volumes ?

\section{Return}

\subsection{Depended Variables}
 \textbf{Close to Open} return is calculated using
the closing price at the event day and the open price on the next day.

 \textbf{Open to Close} return is calculated using the open price and closing price on
the next day.

 \textbf{Forward return}  is day 1,2, 3, 4, 5 and so on returns from the event day.

\subsection{Control Variables}
\textbf{Upper hit} and \textbf{Lower hit} are dummy variables that indicates upper and lower limit touched at time t.

\textbf{Upper} and \textbf{Lower} are dummy variables that indicates maximum or minimum trading price is above and lower than half of daily limit.

\textbf{Middle} is dummy variable that indicates maximum and minimum trading price is lower than half of daily limit.

\textbf{Limit Change} is dummy variable that indicated changing in price limit at time t.


\newgeometry{top=15mm, bottom=25mm,left = 15mm,right = 15mm}
\begin{landscape}
\begin{table}[htbp]
\centering
\lr{
\def\sym#1{\ifmmode^{#1}\else\(^{#1}\)\fi}
\begin{tabular}{l*{8}{c}}
\hline\hline
            &\multicolumn{1}{c}{(1)}&\multicolumn{1}{c}{(2)}&\multicolumn{1}{c}{(3)}&\multicolumn{1}{c}{(4)}&\multicolumn{1}{c}{(5)}&\multicolumn{1}{c}{(6)}&\multicolumn{1}{c}{(7)}&\multicolumn{1}{c}{(8)}\\
            &\multicolumn{1}{c}{closetoopen}&\multicolumn{1}{c}{opentoclose}&\multicolumn{1}{c}{ret\_1}&\multicolumn{1}{c}{ret\_2}&\multicolumn{1}{c}{ret\_2to5}&\multicolumn{1}{c}{ret\_5to50}&\multicolumn{1}{c}{ret\_50to100}&\multicolumn{1}{c}{ret\_100to300}\\
\hline
upperhit    &       3.487\sym{***}&      -1.139\sym{***}&       4.823\sym{***}&       7.806\sym{***}&       5.625\sym{***}&       11.02         &      -53.13\sym{**} &      1806.9         \\
            &     (15.39)         &    (-20.94)         &      (6.02)         &      (8.71)         &      (8.89)         &      (1.51)         &     (-2.60)         &      (1.31)         \\
[1em]
upper       &       1.595\sym{***}&      -0.820\sym{***}&       2.068\sym{***}&       3.142\sym{***}&       2.145\sym{***}&      -2.541         &      -49.55\sym{**} &     -1503.6         \\
            &      (8.03)         &    (-17.73)         &      (5.23)         &      (7.07)         &      (5.72)         &     (-0.40)         &     (-2.97)         &     (-1.34)         \\
[1em]
middle      &      -1.327\sym{***}&      -0.341\sym{***}&       0.962\sym{*}  &       2.120\sym{***}&       2.196\sym{***}&       49.85\sym{*}  &       229.5\sym{***}&       244.0         \\
            &     (-5.07)         &     (-6.71)         &      (2.10)         &      (4.06)         &      (5.77)         &      (2.38)         &      (4.23)         &      (0.19)         \\
[1em]
lower       &      -1.115\sym{***}&     -0.0994\sym{**} &      -0.659\sym{*}  &       0.625         &       4.211\sym{***}&       18.94\sym{***}&       56.53\sym{***}&       151.2         \\
            &     (-6.65)         &     (-2.63)         &     (-2.58)         &      (1.74)         &      (8.24)         &      (3.48)         &      (3.77)         &      (0.14)         \\
[1em]
lowerhit    &      -2.924\sym{***}&       0.488\sym{***}&      -3.181\sym{***}&      -3.064\sym{***}&       3.478\sym{***}&       31.58\sym{***}&       14.41         &      1911.6         \\
            &    (-10.72)         &      (5.67)         &     (-4.68)         &     (-4.03)         &      (6.99)         &      (7.88)         &      (1.10)         &      (1.42)         \\
[1em]
limitchange &       7.524\sym{***}&     -0.0248         &       11.07         &       16.15         &       0.797         &      -244.1\sym{*}  &     -1088.1\sym{***}&     16708.8\sym{***}\\
            &      (9.33)         &     (-0.37)         &      (1.38)         &      (1.90)         &      (0.80)         &     (-2.23)         &     (-4.15)         &      (3.81)         \\
\hline
\(N\)       &      119863         &      114831         &      119863         &      119598         &      118628         &      106075         &      101503         &       70705         \\
\hline\hline
\multicolumn{9}{l}{\footnotesize \textit{t} statistics in parentheses}\\
\multicolumn{9}{l}{\footnotesize We controll for marketratio and limitgroup}\\
\multicolumn{9}{l}{\footnotesize \sym{*} \(p<0.05\), \sym{**} \(p<0.01\), \sym{***} \(p<0.001\)}\\
\end{tabular}
}

\caption{OLS regression, Clustered by calendar date}
\end{table}


\begin{table}[htbp]
\centering
\lr{
\def\sym#1{\ifmmode^{#1}\else\(^{#1}\)\fi}
\begin{tabular}{l*{8}{c}}
\hline\hline
            &\multicolumn{1}{c}{(1)}&\multicolumn{1}{c}{(2)}&\multicolumn{1}{c}{(3)}&\multicolumn{1}{c}{(4)}&\multicolumn{1}{c}{(5)}&\multicolumn{1}{c}{(6)}&\multicolumn{1}{c}{(7)}&\multicolumn{1}{c}{(8)}\\
            &\multicolumn{1}{c}{closetoopen}&\multicolumn{1}{c}{opentoclose}&\multicolumn{1}{c}{ret\_1}&\multicolumn{1}{c}{ret\_2}&\multicolumn{1}{c}{ret\_2to5}&\multicolumn{1}{c}{ret\_5to50}&\multicolumn{1}{c}{ret\_50to100}&\multicolumn{1}{c}{ret\_100to300}\\
\hline
upperhit    &       3.487\sym{***}&      -1.139\sym{***}&       4.823\sym{***}&       7.806\sym{***}&       5.625\sym{***}&       11.02         &      -53.13         &      1806.9         \\
            &     (12.83)         &    (-22.84)         &      (6.00)         &      (8.55)         &      (6.21)         &      (0.71)         &     (-0.74)         &      (1.03)         \\
[1em]
upper       &       1.595\sym{***}&      -0.820\sym{***}&       2.068\sym{***}&       3.142\sym{***}&       2.145\sym{***}&      -2.541         &      -49.55         &     -1503.6         \\
            &      (7.07)         &    (-19.30)         &      (5.25)         &      (7.30)         &      (4.65)         &     (-0.19)         &     (-0.82)         &     (-0.98)         \\
[1em]
middle      &      -1.327\sym{***}&      -0.341\sym{***}&       0.962\sym{*}  &       2.120\sym{***}&       2.196\sym{***}&       49.85         &       229.5         &       244.0         \\
            &     (-4.97)         &     (-7.40)         &      (2.07)         &      (4.09)         &      (4.19)         &      (1.03)         &      (1.02)         &      (0.73)         \\
[1em]
lower       &      -1.115\sym{***}&     -0.0994\sym{**} &      -0.659\sym{**} &       0.625         &       4.211\sym{***}&       18.94         &       56.53         &       151.2         \\
            &     (-5.18)         &     (-3.05)         &     (-2.70)         &      (1.61)         &      (5.82)         &      (1.74)         &      (1.16)         &      (0.73)         \\
[1em]
lowerhit    &      -2.924\sym{***}&       0.488\sym{***}&      -3.181\sym{***}&      -3.064\sym{***}&       3.478\sym{***}&       31.58\sym{***}&       14.41         &      1911.6         \\
            &     (-9.57)         &     (12.21)         &     (-4.76)         &     (-4.13)         &      (5.60)         &      (6.19)         &      (1.16)         &      (1.02)         \\
[1em]
limitchange &       7.524\sym{***}&     -0.0248         &       11.07         &       16.15         &       0.797         &      -244.1         &     -1088.1         &     16708.8         \\
            &     (19.02)         &     (-0.43)         &      (1.39)         &      (1.91)         &      (0.96)         &     (-0.94)         &     (-0.97)         &      (1.01)         \\
\hline
\(N\)       &      119863         &      114831         &      119863         &      119598         &      118628         &      106075         &      101503         &       70705         \\
\hline\hline
\multicolumn{9}{l}{\footnotesize \textit{t} statistics in parentheses}\\
\multicolumn{9}{l}{\footnotesize We controll for marketratio and limitgroup}\\
\multicolumn{9}{l}{\footnotesize \sym{*} \(p<0.05\), \sym{**} \(p<0.01\), \sym{***} \(p<0.001\)}\\
\end{tabular}
}

\caption{OLS regression, Clustered by calendar Stocks}
\end{table}

\begin{table}[htbp]
\centering
\lr{
\def\sym#1{\ifmmode^{#1}\else\(^{#1}\)\fi}
\begin{tabular}{l*{8}{c}}
\hline\hline
            &\multicolumn{1}{c}{(1)}&\multicolumn{1}{c}{(2)}&\multicolumn{1}{c}{(3)}&\multicolumn{1}{c}{(4)}&\multicolumn{1}{c}{(5)}&\multicolumn{1}{c}{(6)}&\multicolumn{1}{c}{(7)}&\multicolumn{1}{c}{(8)}\\
            &\multicolumn{1}{c}{closetoopen}&\multicolumn{1}{c}{opentoclose}&\multicolumn{1}{c}{ret\_1}&\multicolumn{1}{c}{ret\_2}&\multicolumn{1}{c}{ret\_2to5}&\multicolumn{1}{c}{ret\_5to50}&\multicolumn{1}{c}{ret\_50to100}&\multicolumn{1}{c}{ret\_100to300}\\
\hline
upperhit    &       4.674\sym{***}&      -1.122\sym{***}&       4.150\sym{***}&       6.669\sym{***}&       4.068\sym{***}&      -29.48         &      -209.3         &       54.11         \\
            &     (17.26)         &    (-22.40)         &      (3.42)         &      (5.22)         &      (6.52)         &     (-0.61)         &     (-0.97)         &      (0.30)         \\
[1em]
upper       &       1.998\sym{***}&      -0.786\sym{***}&       1.962\sym{***}&       2.882\sym{***}&       1.600\sym{***}&       0.660         &      -29.07         &      -830.5         \\
            &      (8.56)         &    (-18.70)         &      (5.14)         &      (7.09)         &      (4.07)         &      (0.08)         &     (-0.77)         &     (-0.97)         \\
[1em]
middle      &      -0.299         &      -0.322\sym{***}&       0.624\sym{*}  &       1.486\sym{***}&       1.452\sym{***}&       42.67         &       211.8         &     -3666.6         \\
            &     (-1.26)         &     (-7.37)         &      (2.55)         &      (5.03)         &      (3.73)         &      (1.00)         &      (1.02)         &     (-1.01)         \\
[1em]
lower       &      0.0751         &      -0.101\sym{**} &      -0.948\sym{**} &      0.0277         &       3.219\sym{***}&       14.41         &       45.68         &      -641.0         \\
            &      (0.41)         &     (-3.23)         &     (-3.04)         &      (0.07)         &      (5.84)         &      (1.39)         &      (1.03)         &     (-0.99)         \\
[1em]
lowerhit    &      -1.159\sym{***}&       0.467\sym{***}&      -3.512\sym{***}&      -3.656\sym{***}&       2.718\sym{***}&       28.31\sym{***}&       10.20         &      1510.9         \\
            &     (-5.79)         &     (12.14)         &     (-4.33)         &     (-4.21)         &      (5.51)         &      (5.23)         &      (0.50)         &      (1.02)         \\
[1em]
limitchange &       5.481\sym{***}&    -0.00359         &       9.134         &       14.37         &     -0.0678         &      -437.0         &     -1908.2         &     14898.9         \\
            &     (15.38)         &     (-0.07)         &      (1.32)         &      (1.95)         &     (-0.08)         &     (-1.01)         &     (-1.00)         &      (1.01)         \\
\hline
\(N\)       &      119863         &      114831         &      119863         &      119598         &      118628         &      106075         &      101503         &       70705         \\
\hline\hline
\multicolumn{9}{l}{\footnotesize \textit{t} statistics in parentheses}\\
\multicolumn{9}{l}{\footnotesize We controll for marketratio and limitgroup}\\
\multicolumn{9}{l}{\footnotesize \sym{*} \(p<0.05\), \sym{**} \(p<0.01\), \sym{***} \(p<0.001\)}\\
\end{tabular}
}

\caption{Fixed Effect regression on stocks}
\end{table}
\end{landscape}


\FloatBarrier
\restoregeometry
\section{Volume}

\subsection{Depended Variables}

$ \textbf{Turn}_{k,t} $ Stock k's turnover on date t is defined as the amount traded in Rial divided by the market capitalization of the free float.

\begin{equation}
\text{Turn}_{k,t} = \frac{\text{Volume}(\text{Rial})_{k,t}}{\text{MarketCap(FreeFloat)}_{k,t}}
\end{equation}

\textbf{Relative turnover} is defined as the ratio of the turnover of stock k on date t to the average turnover of stock k during our sample period.

\begin{equation}
\text{RelTurn}_{k,t} = \frac{\text{Turn}_{k,t}}{AVG(\text{Turn}_{k,t})}
\end{equation}



\newgeometry{top=15mm, bottom=35mm,left = 15mm,right = 15mm}

\begin{table}[htbp]
\centering
{
\def\sym#1{\ifmmode^{#1}\else\(^{#1}\)\fi}
\begin{tabular}{l*{3}{c}}
\hline\hline
            &\multicolumn{1}{c}{(1)}&\multicolumn{1}{c}{(2)}&\multicolumn{1}{c}{(3)}\\
            &\multicolumn{1}{c}{lnvolume}&\multicolumn{1}{c}{turn}&\multicolumn{1}{c}{relturn}\\
\hline
upperhit    &       0.908\sym{***}&      0.0313\sym{*}  &       1.230\sym{***}\\
            &     (17.33)         &      (2.48)         &     (25.89)         \\
[1em]
upper       &       0.263\sym{***}&     0.00959\sym{*}  &       0.317\sym{***}\\
            &      (6.31)         &      (2.00)         &     (10.14)         \\
[1em]
middle      &      -0.990\sym{***}&    -0.00531\sym{***}&      -0.357\sym{***}\\
            &    (-21.73)         &     (-5.03)         &     (-7.33)         \\
[1em]
lower       &      -0.748\sym{***}&     0.00927         &      -0.160\sym{***}\\
            &    (-21.94)         &      (0.78)         &     (-3.96)         \\
[1em]
lowerhit    &      -0.176\sym{*}  &      0.0155\sym{*}  &       0.449\sym{***}\\
            &     (-2.54)         &      (2.10)         &      (9.71)         \\
[1em]
limitchange &       0.223\sym{*}  &      0.0418         &       0.246\sym{*}  \\
            &      (2.33)         &      (1.12)         &      (2.47)         \\
\hline
\(N\)       &      119863         &      119863         &      119863         \\
\hline\hline
\multicolumn{4}{l}{\footnotesize \textit{t} statistics in parentheses}\\
\multicolumn{4}{l}{\footnotesize We controll for marketratio and limitgroup}\\
\multicolumn{4}{l}{\footnotesize \sym{*} \(p<0.05\), \sym{**} \(p<0.01\), \sym{***} \(p<0.001\)}\\
\end{tabular}
}

\caption{OLS regression, Clustered by calendar date}
\end{table}

\begin{table}[htbp]
\centering
\lr{
\def\sym#1{\ifmmode^{#1}\else\(^{#1}\)\fi}
\begin{tabular}{l*{6}{c}}
\hline\hline
            &  Cluster(t)         &                     &                     &          FE         &                     &                     \\
            &\multicolumn{1}{c}{(1)}&\multicolumn{1}{c}{(2)}&\multicolumn{1}{c}{(3)}&\multicolumn{1}{c}{(4)}&\multicolumn{1}{c}{(5)}&\multicolumn{1}{c}{(6)}\\
            &\multicolumn{1}{c}{lnvolume}&\multicolumn{1}{c}{turn}&\multicolumn{1}{c}{relturn}&\multicolumn{1}{c}{lnvolume}&\multicolumn{1}{c}{turn}&\multicolumn{1}{c}{relturn}\\
\hline
upperhit    &       0.908\sym{***}&      0.0313\sym{*}  &       1.230\sym{***}&       1.328\sym{***}&      0.0178\sym{***}&       1.330\sym{***}\\
            &     (17.33)         &      (2.48)         &     (25.89)         &     (30.27)         &     (18.49)         &     (31.79)         \\
[1em]
upper       &       0.263\sym{***}&     0.00959\sym{*}  &       0.317\sym{***}&       0.528\sym{***}&     0.00375\sym{***}&       0.360\sym{***}\\
            &      (6.31)         &      (2.00)         &     (10.14)         &     (16.65)         &      (3.44)         &     (11.98)         \\
[1em]
middle      &      -0.990\sym{***}&    -0.00531\sym{***}&      -0.357\sym{***}&      -0.634\sym{***}&     -0.0102         &      -0.358\sym{***}\\
            &    (-21.73)         &     (-5.03)         &     (-7.33)         &    (-19.63)         &     (-1.65)         &     (-6.10)         \\
[1em]
lower       &      -0.748\sym{***}&     0.00927         &      -0.160\sym{***}&      -0.189\sym{***}&   0.0000959         &      -0.111\sym{**} \\
            &    (-21.94)         &      (0.78)         &     (-3.96)         &     (-8.24)         &      (0.06)         &     (-2.75)         \\
[1em]
lowerhit    &      -0.176\sym{*}  &      0.0155\sym{*}  &       0.449\sym{***}&       0.518\sym{***}&     0.00898\sym{***}&       0.557\sym{***}\\
            &     (-2.54)         &      (2.10)         &      (9.71)         &     (13.67)         &      (7.94)         &      (6.80)         \\
[1em]
limitchange &       0.223\sym{*}  &      0.0418         &       0.246\sym{*}  &      -0.104\sym{**} &      0.0171         &       0.209         \\
            &      (2.33)         &      (1.12)         &      (2.47)         &     (-3.24)         &      (1.09)         &      (1.87)         \\
\hline
\(N\)       &      119863         &      119863         &      119863         &      119863         &      119863         &      119863         \\
\hline\hline
\multicolumn{7}{l}{\footnotesize \textit{t} statistics in parentheses}\\
\multicolumn{7}{l}{\footnotesize We controll for marketratio and limitgroup}\\
\multicolumn{7}{l}{\footnotesize \sym{*} \(p<0.05\), \sym{**} \(p<0.01\), \sym{***} \(p<0.001\)}\\
\end{tabular}
}

\caption{Fixed Effect regression on stocks}
\end{table}
\end{document}